\documentclass[a4paper]{article}
\usepackage[left=3cm, right=3cm]{geometry}
\usepackage{lipsum}
\usepackage{tikzpagenodes}
\usepackage{pgfplots}
\usepackage{tikz}
\usepackage{tikz-3dplot}
\usetikzlibrary{arrows,decorations.pathmorphing,backgrounds,positioning,fit,matrix}
\pgfplotsset{compat=1.8}
\usepackage{graphics} % for pdf, bitmapped graphics files
\usepackage{epsfig} % for postscript graphics files
\usepackage[colorlinks=true,citecolor=green]{hyperref}
\usepackage{cite}
\usepackage{amsmath,amssymb,amsfonts}
\usepackage{algorithmic}
\usepackage{graphicx}
\usepackage{url}
\usepackage{cite}
\usepackage{bm}
\usepackage{pbox}
\usepackage{siunitx,booktabs,etoolbox}
\usepackage{ulem}
\usepackage{titling}
\usepackage{float}
%\usepackage{pgf,tikz,pgfplots}
%\pgfplotsset{compat=1.15}
%\usepackage{mathrsfs}

\usetikzlibrary{arrows}

\def\BibTeX{{\rm B\kern-.05em{\sc i\kern-.025em b}\kern-.08em
    T\kern-.1667em\lower.7ex\hbox{E}\kern-.125emX}}
\title{Ex1: Introductory exercise}
\author{Xiao Hu, emails: \url{xiahaa@space.dtu.dk}}
\begin{document}
	\maketitle
	\thispagestyle{empty}
%\maketitle
%\clearpage
\section{Results}
Results are shown as follows:
\begin{figure}[H]
\centering
\includegraphics[width=0.7\textwidth]{./figures/a1.png}
\caption{Total Variation (TV) of raw image: 119.5585; TV of smoothed image: 40.9807.}
\end{figure}
%\subsection{Q2}
\begin{figure}[H]
	\centering
	\includegraphics[width=0.7\textwidth]{./figures/a2.png}
	\caption{Counting boundary length (Only boundary with length over 100 are shown with a textbox). The number in each textbox indicate the length of the corresponding boundary.}
\end{figure}
%\subsection{Q3}
\begin{figure}[H]
	\centering
	\includegraphics[width=0.8\textwidth]{./figures/a3.png}
	\caption{Curve smoothing: comparison of three different strategies of curve smoothing.}
\end{figure}
\begin{figure}[H]
	\centering
	\includegraphics[width=0.6\textwidth]{./figures/a4.png}
	\caption{Image warpping (see complementary Gif figure for an animation).}
\end{figure}
\begin{figure}[H]
	\centering
	\includegraphics[width=0.6\textwidth]{./figures/a5.png}
	\caption{Volumetric Image.}
\end{figure}

\begin{figure}[H]
	\centering
	\includegraphics[width=0.6\textwidth]{./figures/a6.png}
	\caption{PCA of multispectral image. The first three channels are used to generate this RGB image. The maximum absolute difference of my implementation and the \textsc{Matlab} internal function is $1.1235e-11$.}
\end{figure}



\begin{figure}
	\centering
	\includegraphics[width=0.9\textwidth]{./figures/a7.png}
	\caption{Bacterial growth from movie frames. This is a snapshot of a single frame. See the complementary Gif for the animation.}
	\includegraphics[width=0.7\textwidth]{./figures/a72}
	\caption{Bacterial growth curve: assume one pixel equals to one bacteria.}
\end{figure}

%\bibliography{PnPCites} 
%\bibliographystyle{ieeetr}

\end{document}